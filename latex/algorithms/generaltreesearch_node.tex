\begin{algorithm}[H]
    \caption{Data structures}
    \label{alg:general_tree_search_data}
    \begin{algorithmic}[1]
            \Obj{Node}
            \State \textbf{private} State: \textproc{GameState}
            \State \textbf{private} Parent: \textproc{Node} \textbf{or} None
            \State \textbf{private} Children: \textproc{Node}[ ]
            \State \textbf{private} Values = \{\}
            \Comment Mutable fields, stores e.g. evaluation, cumulative utility, count, etc.
            \State \textbf{private} OnFrontier = True            
            \Comment New nodes start on the frontier
            \State \textbf{private} GeneratingAction
            \State \textbf{private} UnexpandedActions
            \\
            \State $\rhd$ Most node attributes are read-only by default
            \State \textbf{public} getState()
            \State \textbf{public} getValues()
            \State \textbf{public} getParentValues()
            \State \textbf{public} getChildrenValues()
            \State \textbf{public} getGeneratingAction()
            \State \textbf{public} popUnexpandedActions()
        \EndObj
        \\
        \State $\rhd$ These functions allow access to otherwise private node attributes
        \State \textbf{function} \textproc{GetParent}(Node: \textproc{Node})
        \State \textbf{function} \textproc{GetChildren}(Node: \textproc{Node})
        \State \textbf{function} \textproc{AddChild}(Node: \textproc{Node}, Child: \textproc{Node})
        \State \textbf{function} \textproc{DelChild}(Node: \textproc{Node}, Child: \textproc{Node})
        \State \textbf{function} \textproc{SetValues}(Node: \textproc{Node})
    \end{algorithmic}
    \vspace{.25cm}
    \begin{algorithmic}[1]
        \Obj{Frontier}
            \State \textbf{private} Nodes: \textproc{Node}[ ]
            \Comment Ordered list, e.g. LIFO, FIFO or Prioity Queue
            \State \textbf{public} popFrontier()
            \Comment Remove and return last node in Nodes
            \EndObj
    \end{algorithmic}
    \vspace{.25cm}
    \begin{algorithmic}[1]
        \Obj{Parameters}
            \State \textbf{public} Data = \{ \}
            \Comment Stores algorithm parameters, e.g. maximum expansion depth
        \EndObj
    \end{algorithmic}
\end{algorithm}