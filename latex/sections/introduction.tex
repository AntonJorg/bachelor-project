\section{Introduction}

Ever since the infancy of the artificial intelligence field, search
algorithms have been a key part of designing rational agents. By
looking many moves ahead agents can review many lines of play, and
compare evaluations of millions of future states deep into the search
tree to make much better decisions than an agent that simply evaluates
the immediate consequences of its actions.

In his 1928 paper 'Zur Theorie der Gesellschaftsspiele' von Neumann
proves the minimax theorem which shows that
\begin{equation}
    \max_{x\in X}\min_{y\in Y} f(x, y) = \min_{y\in Y}\max_{x\in X} f(x, y)   
\end{equation}
holds under certain conditions. These conditions notably apply in the
case of simultaneous games, where the expexted value of a game can be
calculated using $x^\intercal Ay$. With this he proves that every zero-sum two-player game has
optimal mixed strategies, which he himself states as being a prerequisite
for the entire field of game theory. 

Alan Turing and David Champernowne developed the Turochamp chess program
in 1948, which is one of the earliest examples of a minimax algorithm.
Turochamp examines all of its own possible moves, and for each of those
moves it examines every possible player response. If any 'notable' moves
like checks, captures of undefended pieces, etc. are available, it
examines these as well. All resulting positions are evaluated, and these
evaluations were backpropagated using minimax. 

During the famous Dartmouth workshop in 1956 the topic of minimax was
on the agenda, and John McCarthy who arranged the workshop is said to 
have suggested using alpha beta pruning to his students working on a 
chess program in 1961.

Even with the huge improvements from alpha beta pruning traditional
tree search methods struggled with games with a high branching factor,
of wich Go is the most popular one in AI research. This lack of
performance led researchers to seek different methods, and in 2006 
Rémi Coulom introduced Monte Carlo Tree Search, extending
traditional Monte Carlo methods to tree search.

Many mixed algoithms using both minimax and mcts 

two different paradigms, many similarities.