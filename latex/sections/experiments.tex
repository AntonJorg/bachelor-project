\section{Experiments}
\subsection{Agents}
\subsubsection{MiniMax Based}
\begin{itemize}
    \item Iterative Deepening : ID
    \item Iterative Deepening w/ $\alpha\beta$-pruning: ID-AB
    \item Iterative Deepening w/ Simulation : ID-Sim
    \item Beam Search Agent: BS
    \item Best First MiniMax: BFMM
    \item MiniMax From MCTS Tree: MM-MCTS-T
\end{itemize}

\subsubsection{MCTS Based}
\begin{itemize}
    \item Monte Carlo Tree Search: MCTS
    \item MCTS Static Rollout: MCTS-SR
    \item Partial Expansion: MCTS-PE
    \item Static Weighted MCTS: MCTS-SW
    \item MiniMax Weighted MCTS: MCTS-MMW
    \item Progressive Pruning MCTS: PP-MCTS
\end{itemize}

\subsection{Environment: Connect Four}
Connect Four has been chosen for the following reasons:
It is simple to implement
Utility can be calculated quickly
It is a solved game, which makes it possible
to compare agent behavior to optimal play.


\subsection{Comparison}

Firstly, the agents within each group will play against each other
such that each agent type plays against all other agent types, as well
as against itself. Since both groups are of size 6, this will result
in $2n \cdot 6^2 = 72n$ games, where $n$ is the number of games in
each matchup.

From each group, the agents that significantly outperform all others
will be picked out, and another round of games will be run.

Finally, the top performing agents will be pitted against a recursive
NegaMax based agent, to demonstrate the performance loss MiniMax suffers
from being converted to an iterative function.