\section{Generalizing search algorithms}

It is clear from Section \ref{sec:theory} that MiniMax with its various
enhancements along with MCTS are powerful decision making tools in an 
adversarial setting. What is not clear is that the two algorithms are more
similar than it initially seems. The goal of this section is to introduce 
a more general framework in pseudocode that lets both algorithms be 
implemented, while still describing the structure of the algorithms, eg.
\lstinline|TreeSearch(state): return GoodMove| would be very general, but
not terribly informative.

Initially the focus will be on the standard MiniMax algorithm (Algorithm 
\ref{alg:minimax}) and MCTS (Algorithm \ref{alg:mcts}), but later the 
framework will be extended to handle all the improvements described in 
Section \ref{sec:theory} as well, which will allow it to describe a much 
broader class of tree search algorithms.

The perhaps most glaring incompatibility between the two algorithms as
presented, is that MiniMax is recursive while MCTS is iterative. To
arrive at a common pseudocode one of them must therefore be converted
to the others type. It is generally possible to convert any recursive
computation to an iterative one, since the recursive computation is
implemented using a call stack, which the program then iterates 
over. Using a stack is common in many non-adversarial search algorithms
to represent the frontier, so converting recursive minimax to a 
frontier based algorithm is the choice made in this paper. 
The recursive definition from 
Algorithm \ref{alg:minimax} clearly searches nodes in a depth first
manner; no state can be evaluated before all its successor states
have. The tried and tested way to implement DFS
iteratively is a Last In First Out (LIFO) queue as frontier, originally populated
only with the root of the tree. Each iteration a node $n$ is popped from the
frontier, and a child node is generated and pushed to the frontier for each
successor state of $n.state$. If $n.state$ has no successors it must be
a terminal state, and $U(n.state)$ can be computed. This utility value
can then be backpropagated up through the tree by going to the parent node
$p$, and assigning the min or max of its childrens utilities to itself, 
depending on $Player(p.state)$. This backpropagation continues until
a node is reached where one ore more of its children have not had a value
assigned yet. When the utility of all leaf nodes have been computed the
backpropagation reaches all the way to the root node, and the MiniMax move
can be calculated.

\begin{algorithm}[H]
    \caption{Iterative MiniMax search}
    \label{alg:iterative_minimax}
    \begin{algorithmic}[1]
    
    \Procedure{MiniMax}{$s$}
        \State Frontier $\leftarrow$ LIFO Queue
        \State Root $\leftarrow$ TreeNode($s$)
        \State Frontier.Push(Root)
        \While{Frontier \textbf{not} empty}
            \State Node $\leftarrow$ Frontier.Pop() \Comment{Select node}
            \For{$a \in$ Node.State.ApplicableActions} \Comment{Expand node}
                \State Leaf $\leftarrow$ Node.AddChild(Result(Node.State, $a$))
            \EndFor
            \If{Leaf \textbf{is} None}
                Leaf $\leftarrow$ Node
            \EndIf
            \If{Leaf.State $\in S^\circ$}
                \State Value $\leftarrow U(\text{Leaf.State})$ \Comment{Evaluate node}
                \State Backpropagate(Leaf, Value) \Comment{Backpropagate}
            \EndIf
        \EndWhile
        \State \Return $\arg\max_{\text{C} \in \text{Root.Children}}$ C.Utility
    \EndProcedure
    \end{algorithmic}
\end{algorithm}

When looking at the iterative MiniMax in Algorithm \ref{alg:iterative_minimax},
the similarities with MCTS become more obivious. There is a selection step,
an expansion step, and a backpropagation step. The only thing missing is 
the simulation step, but if we think of simulation as evaluating the state
via a non-deterministic function, all the steps are present. The MiniMax
algorithm includes an if-statement on line 9 that controls whether evaluation
and backpropagation happens, but that too can be abstracted into a
\lstinline|ShouldEvaluate| function, which in the case of MCTS always
returns true. Following these observations, a General Tree Search (GTS)
algorithm that describes both MiniMax and MCTS is formulated:

\begin{algorithm}[H]
    \caption{General Tree Search (GTS)}
    \label{alg:general_tree_search_1}
    \begin{algorithmic}[1]
    
    \Procedure{TreeSearch}{$s$}
        \State Frontier $\leftarrow$ Queue
        \State Root $\leftarrow$ TreeNode($s$)
        \State Frontier.Push(Root)
        \While{\textbf{not} ShouldTerminate()}
            \State Node $\leftarrow$ Select(Root, Frontier)
            \State Leaf $\leftarrow$ Expand(Node)
            \If{ShouldEvaluate(Leaf)}
                \State Value $\leftarrow$ Evaluate(Leaf)
                \State Backpropagate(Leaf, Value)
            \EndIf
        \EndWhile
        \State \Return GetBestMove(Root)
    \EndProcedure
    \end{algorithmic}
\end{algorithm}

Given the GTS framework from Algoritm \ref{alg:general_tree_search_1}, 
MiniMax and MCTS can each be described simply by specifying the 7 
\textit{component functions} that constitute the framework: 
\textproc{Select}, \textproc{Expand}, \textproc{ShouldEvaluate}, 
\textproc{Evaluate}, \textproc{Backpropagate}, and \textproc{GetBestMove}.

\begin{table}[H]
    \centering
    \begin{tabular}{|l|r|}
        \hline
        \multicolumn{2}{|c|}{MiniMax} \\ \hline
        ShouldTerminate & WhenRootHasValue \\ \hline
        Select & DFSSelect \\ \hline
        Expand & ExpandAll \\ \hline
        ShouldEvaluate & IfStateIsTerminal \\ \hline
        Evaluate & Utility \\ \hline
        Backpropagate & BackpropagateMiniMax \\ \hline
        GetBestMove & MiniMaxMove \\ \hline
    \end{tabular}
    \quad
    \begin{tabular}{|l|r|}
        \hline
        \multicolumn{2}{|c|}{MCTS} \\ \hline
        ShouldTerminate & WhenBudgetExceeded \\ \hline
        Select & UCTSelect \\ \hline
        Expand & ExpandNext \\ \hline
        ShouldEvaluate & Always \\ \hline
        Evaluate & Simulate \\ \hline
        Backpropagate & BackpropagateSum \\ \hline
        GetBestMove & MostRobustChild \\ \hline
    \end{tabular}
    \caption{MiniMax and MCTS as implementations of GTS. Pseudocode
    for the component functions can be found in Appendix \todo}
    \label{tab:minimax_mcts_schematics}
\end{table}

In order for such a framework to be descriptive at all some limitations
must be put on the component functions: 

While this is certainly a big step forward, GTS lacks some key features
that makes it impossible to implement much more than the basic versions
of MiniMax and MCTS, and both have to do with pruning. First off, in 
GTS it is only possible to call \bp together with a call to \eval. This
makes it difficult to implement techniques such as $\alpha\beta$-pruning,
where expansion of a node stops if the last child attains a value outside
the $\alpha\beta$ bounds. In this case \expand returns the input node,
but this node does not need to be evaluated, it just needs to
backpropagate values from its children. To alleviate this issue an
additional component function \shbp is introduced, and \bp is now
conditional on that instead of \sheval.

Secondly the search tree in GTS can only grow, since none of the
component functions are allowed to remove nodes from the tree. This
excludes a wide class of search algorithms such as iterative deepening
and MCTS with retroactive pruning of unpromising nodes. All the
algorithm needs is a component function that lets it remove nodes
from the search tree, and for this \reflect is introduced as the
last step in the main loop. With these modifications, the Extended
General Tree Search (EGTS) looks like this:

\begin{algorithm}[H]
    \caption{Extended General Tree Search (EGTS)}
    \label{alg:general_tree_search_2}
    \begin{algorithmic}[1]
    
    \Procedure{TreeSearch}{$s$}
        \State Frontier $\leftarrow$ Queue
        \State Root $\leftarrow$ TreeNode($s$)
        \State Frontier.Push(Root)
        \While{\textbf{not} ShouldTerminate()}
            \State Node $\leftarrow$ Select(Root, Frontier)
            \State Leaf $\leftarrow$ Expand(Node)
            \State Value $\leftarrow$ None
            \If{ShouldEvaluate(Leaf)}
                \State Value $\leftarrow$ Evaluate(Leaf)
            \EndIf
            \If{ShouldBackpropagate(Leaf, Value)}
                \State Backpropagate(Leaf, Value)
            \EndIf
            \State Reflect(Root)
        \EndWhile
        \State \Return GetBestMove(Root)
    \EndProcedure
    \end{algorithmic}
\end{algorithm}

% \begin{figure}[H]
    \centering
    
    % cannot make tikzset work for some reason
    \tikzstyle{startstop} = [rectangle, rounded corners, minimum width=3cm, minimum height=1cm,text centered, draw=black, fill=red!30]
    \tikzstyle{io} = [trapezium, trapezium left angle=70, trapezium right angle=110, minimum width=3cm, minimum height=1cm, text centered, inner xsep=0pt,outer sep=0pt, draw=black, fill=blue!30]
    \tikzstyle{process} = [rectangle, minimum width=3cm, minimum height=1cm, text centered, text width=3cm, draw=black, fill=orange!30]
    \tikzstyle{decision} = [diamond, minimum width=3cm, minimum height=1cm, text centered, draw=black, fill=green!30]
    \tikzstyle{arrow} = [thick,->,>=stealth]
    
    
    \begin{tikzpicture}[node distance=2cm]
        \node (start) [startstop] {TreeSearch};
        \node (state) [io, below of=start] {State};
        \node (terminate) [decision, below of=state, yshift=-0.5cm] {Terminate?};
        \node (bestmove) [io, right of=terminate, xshift=3cm] {Get Best Move};
        \node (select) [process, below of=terminate, yshift=-0.5cm] {Select};
        \node (expand) [process, below of=select] {Expand};
        \node (shouldevaluate) [decision, below of=expand, yshift=-0.5cm] {Evaluate?};
        \node (evaluate) [process, below of=shouldevaluate, yshift=-0.5cm] {Evaluate};  
        \node (shouldbp) [decision, below of=evaluate, yshift=-0.5cm] {Backpropagate?};
        \node (bp) [process, below of=shouldbp, yshift=-0.5cm] {Backpropagate};
        \node (reflect) [process, left of=select, xshift=-2cm] {Reflect};
        \node (return) [startstop, below of=bestmove] {Return};

        \draw [arrow] (start) -- (state);
        \draw [arrow] (state) -- (terminate);
        \draw [arrow] (terminate) -- (bestmove);
        \draw [arrow] (terminate) -- (select);
        \draw [arrow] (select) -- (expand);
        \draw [arrow] (expand) -- (shouldevaluate);
        \draw [arrow] (shouldevaluate) -| (reflect);
        \draw [arrow] (shouldevaluate) -- (evaluate);
        \draw [arrow] (evaluate) -- (shouldbp);
        \draw [arrow] (shouldbp) -- (bp);
        \draw [arrow] (shouldbp) -| (reflect);
        \draw [arrow] (bp) -| (reflect);
        \draw [arrow] (reflect) |- (terminate);
        \draw [arrow] (bestmove) -- (return);

        
    \end{tikzpicture}

    \caption{Flowchart of the General Tree Search Algorithm}
    \label{fig:general_algorithm_loop}

\end{figure}


\subsection{Selection and expansion}

A substantial part of any tree search algorithm is how it grows its
search tree, and this is what the selection and expansion functions
are responsible for.

The selection function is the 'compass' of EGTS, and determines where in
the search tree the other functions have to operate. It should be a
pure function with respect to the search tree, such that the tree
and its nodes do not change during the function call. It can however 
change the frontier, and there the selection function should only pop, 
not push.

The expansion function determines how the node returned by Select
grows, and aside from creating child nodes this is the only node that
should be affected by Expand. The child nodes can be pushed to the
frontier, but nothing should be popped.

\subsection{Evaluation and backpropagation}

In order for the selection and expansion steps to be able to
operate efficiently, they need information about the 'quality'
or the expected utility of different branches. Evaluation provides
these estimates, and backpropagation brings that information
up the tree.

The evaluation function should have absolutely no side effects, it
should take a state and return a real number. It can be deterministic
or stochastic.

Backpropagation must only change internal values of nodes, not
tree structure, and it must only propagate up the tree from
children to parents.

\subsection{Reflection}
\todo Find better name, possibly separate pruning and state update

The reflection step adresses the main issue with GTS by allowing
the algorithm to prune nodes from the tree retroactively. The name 
hints at the agent 'reflecting' on the state of the search tree and 
acting accordingly. 

Reflection can only remove nodes from the search tree and frontier,
not add them. Reflection must not change internal values of the nodes.

\subsection{Conditionals}


\subsection{Move selection}
The move selection function is the only function in the algorithm
that is not part of the main loop, and is therefore guaranteed to
run only once. It should base its decision on the search tree, or
on information stored from the reflection step, and it should always
return an action that is applicable in Root.State.
It can be stochastic by drawing from a probability distribution over
the applicable actions.
