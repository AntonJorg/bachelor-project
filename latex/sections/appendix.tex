\section{Component Function Pseudocode}
\label{appendix:componentfunctions}

\subsection*{Selection}
\begin{algorithm}[H]
    \begin{algorithmic}[1]
        \Procedure{DFSSelect}{Root, Frontier}
            \State \Comment Note that expansion functions that do not expand all nodes at once need to place the current node back on the frontier if continued expansion is needed with this select function
            \State \Return Frontier.Pop()
        \EndProcedure
    \end{algorithmic}    
\end{algorithm}

\begin{algorithm}[H]
    \begin{algorithmic}[1]
        \Procedure{PrincipalVariationSelect}{Root, Frontier}
            \If{\textbf{not} IsFullyExpanded(Root)}
                \State \Return Root
            \Else
                \State Children = \textproc{GetChildren}(Root)
                \State EqCh $\gets$ \{C $\in$  Children$|$ C.getValues().Evaluation == Root.getValues().Evaluation\}
                \State Child $\gets$ ChooseRandom(EqCh)
                \State \Return \textproc{PrincipalVariationSelect}(Child, Frontier)
            \EndIf
        \EndProcedure
    \end{algorithmic}    
\end{algorithm}

\begin{algorithm}[H]
    \begin{algorithmic}[1]
        \Procedure{UCTSelect}{Root, Frontier}
            \If{\textbf{not} IsFullyExpanded(Root) \textbf{or} Root.getState() $\in S^\circ$}
                \State \Return Root
            \EndIf
            \State BestChild $\gets$ argmax\{UCT1(C) $|$ C $\in$ \textproc{GetChildren}(Root)\}
            \State \Return \textproc{UCTSelect}(BestChild, Frontier)
        \EndProcedure
    \end{algorithmic}    
\end{algorithm}

\begin{algorithm}[H]
    \begin{algorithmic}[1]
        \Procedure{WeightedSelect}{Root, Frontier}
            \If{\textbf{not} IsFullyExpanded(Root) \textbf{or} Root.getState() $\in S^\circ$}
                \State \Return Root
            \EndIf
            \State N $\gets$ Root.getValues().Count
            \State Max $\gets$ 0
            \State MaximizingChild $\gets$ None
            \For{C $\in$ \textproc{GetChildren}(Root)}
                \State V $\gets$ C.getValues()
                \If{IsMaxNode(Root)}
                    \State Eval $\gets$ V.Evaluation
                    \State Estimate $\gets$ V.CumulativeUtility / V.Count
                \Else 
                    \State Eval = 1 - V.Evaluation
                    \State Estimate $\gets$ 1 - V.CumulativeUtility / V.Count
                \EndIf
                \State Q $\gets$ 1 / $\sqrt{\text{V.Count}}$
                \State W $\gets$ Eval $\cdot$ Q + Estimate $\cdot$ (1 - Q) + $\sqrt{2}\cdot\sqrt{\log(\text{N}) / \text{V.Count}}$
                \If{W $>$ Max}
                    \State Max $\gets$ W
                    \State MaximizingChild $\gets$ C
                \EndIf
            \EndFor 
            \State \Return \textproc{WeightedSelect}(MaximizingChild, Frontier)
        \EndProcedure
    \end{algorithmic}    
\end{algorithm}

\begin{algorithm}[H]
    \begin{algorithmic}[1]
        \Procedure{PartialExpansionSelect}{Root, Frontier}
            \State Children $\gets$ \textproc{GetChildren}(Root)
            \If{Length(Children) == 0 \textbf{or} Root.getState() $\in S^\circ$}
                \State \Return Root
            \EndIf
            \State UCTS $\gets$ \{UCT1(C) $|$ C $\in$ Children\}
            \State UCT$_c$ $\gets$ 0.5 + $\sqrt{2} \cdot \sqrt{\log(\text{Root.getValues().Count}) / (1+\text{Length(Children)})}$
            \If{\textbf{not} IsFullyExpanded(Root) \textbf{and} UCT$_c >$ max(UCTS)}
                \State \Return Root
            \Else
                \State BestChild $\gets$ argmax(UCTS)
                \State \Return \textproc{PartialExpansionSelect}(BestChild, Frontier)
            \EndIf 
        \EndProcedure
    \end{algorithmic}    
\end{algorithm}

\newpage
\subsection*{Expansion}

\begin{algorithm}[H]
    \begin{algorithmic}[1]
        \Procedure{ExpandAll}{Node}
            \While{\textbf{not} IsFullyExpanded(Node)}
                \State Action $\gets$ Node.popUnexpandedActions()
                \State State $\gets$ Result(Node.getState(), Action)
                \State Leaf $\gets$ \textproc{Node}(State)
                \State \textproc{AddChild}(Node, Leaf) 
            \EndWhile
            \State \Return Leaf
        \EndProcedure
    \end{algorithmic}    
\end{algorithm}

\begin{algorithm}[H]
    \begin{algorithmic}[1]
        \Procedure{ExpandNext}{Node}
            \If{IsFullyExpanded(Node)}
                \State \Return Node
            \EndIf
            \State Action $\gets$ Node.popUnexpandedActions()
            \State State $\gets$ Result(Node.getState(), Action)
            \State Leaf $\gets$ \textproc{Node}(State)
            \State \textproc{AddChild}(Node, Leaf) 
            \State \Return Leaf
        \EndProcedure
    \end{algorithmic}    
\end{algorithm}


\begin{algorithm}[H]
    \begin{algorithmic}[1]
        \Procedure{ExpandNextDepthLimited}{Node}
            \If{Node.Values.Depth == Parameters.MaxDepth}
                \State \Return Node
            \EndIf
            \State \Return \textproc{ExpandNext}(Node)
        \EndProcedure
    \end{algorithmic}    
\end{algorithm}

\begin{algorithm}[H]
    \begin{algorithmic}[1]
        \Procedure{ExpandNextAlphaBeta}{Node}
        \State ChildEvals $\gets$ \{V.Evaluation $|$ V $\in$ Node.getChildrenValues()\}
        \If{IsMaxNode(Node) \textbf{and} max(ChildEvals) $>$ Node.Values.Beta}
            \State \Return Node
        \ElsIf{IsMinNode(Node) \textbf{and} min(ChildEvals) $>$ Node.Values.Alpha}
            \State \Return Node
        \EndIf
        \State \Return \textproc{ExpandNextDepthLimited}(Node)
    \EndProcedure
    \end{algorithmic}    
\end{algorithm}

\begin{algorithm}[H]
    \begin{algorithmic}[1]
        \Procedure{ExpandBeam}{Node}
        \State Leaf $\gets$ \textproc{ExpandNextAlphaBeta}(Node)
        \State FilterUnexpandedActions(Leaf)
        \Comment Beam search action filtering happens here
        \State \Return Leaf
        \EndProcedure
    \end{algorithmic}    
\end{algorithm}

\newpage
\subsection*{Evaluation}

\begin{algorithm}[H]
    \begin{algorithmic}[1]
        \Procedure{Utility}{State}
            \State \Return U(State)
        \EndProcedure
    \end{algorithmic}    
\end{algorithm}

\begin{algorithm}[H]
    \begin{algorithmic}[1]
        \Procedure{StaticEvaluation}{State}
            \State \Return E(State)
            \Comment Note that in practice E needs to be implemented for each environment
        \EndProcedure
    \end{algorithmic}    
\end{algorithm}

\begin{algorithm}[H]
    \begin{algorithmic}[1]
        \Procedure{Simulate}{State}
            \While{\textbf{not} State $\in S^\circ$}
                \State Action $\gets$ ChooseRandom(A(State))
                \State State $\gets$ Result(State, Action)
            \EndWhile
            \State \Return U(State)
        \EndProcedure
    \end{algorithmic}    
\end{algorithm}

\begin{algorithm}[H]
    \begin{algorithmic}[1]
        \Procedure{SimulateMany}{State}
            \State Utilities $\gets$ [ ]
            \For{i \textbf{in} \{1, \dots, Parameters.NSimulations\}}
                \State Utilities.Add(\textproc{Simulate}(State))
            \EndFor
            \State \Return Mean(Utilities)
        \EndProcedure
    \end{algorithmic}    
\end{algorithm}

\begin{algorithm}[H]
    \begin{algorithmic}[1]
        \Procedure{StaticAndSimulate}{State}
            \State Evaluation $\gets$ \textproc{StaticEvaluation}(State)
            \State Simulation $\gets$ \textproc{Simulate}(State)
            \State \Return (Evaluation, Simulation)
        \EndProcedure
    \end{algorithmic}    
\end{algorithm}

\newpage
\subsection*{Backpropagation}
\begin{algorithm}[H]
    \begin{algorithmic}[1]
        \Procedure{BackpropagateMiniMax}{Node, Value}
            \Procedure{BP}{Node}
                \State MaxEval $\gets$ max\{V.Evaluation \textbf{for} V $\in$ Node.getChildrenValues()\}
                \State MinEval $\gets$ min\{V.Evaluation \textbf{for} V $\in$ Node.getChildrenValues()\}
                \If{IsMaxNode(Node)}
                    \State Node.Values.Alpha $\gets$ max\{Node.getValues().Alpha, MaxEval\}
                \Else
                    \State Node.Values.Beta $\gets$ min\{Node.getValues().Beta, MinEval\}
                \EndIf
                
                \If{AllChildrenHasValue(Node) \textbf{and} IsFullyExpanded(Node)}
                    \If{IsMaxNode(Node)}
                        \State Node.getValues().Evaluation $\gets$ MaxEval
                    \Else
                        \State Node.getValues().Evaluation $\gets$ MinEval
                    \EndIf
                    \State Parent $\gets$ \textproc{GetParent}(Node)
                    \If{Parent}
                        \textproc{BP}(Parent)
                    \EndIf                    
                \EndIf

            \EndProcedure
            \If{Value \textbf{is} None}
                \State \textproc{BP}(Node)
            \Else
                \State Node.getValues().Evaluation $\gets$ Value
                \State Parent $\gets$ \textproc{GetParent}(Node)
                \If{Parent} 
                    \textproc{BP}(Parent)
                \EndIf
            \EndIf
        \EndProcedure
    \end{algorithmic}    
\end{algorithm}

\begin{algorithm}[H]
    \begin{algorithmic}[1]
        \Procedure{BackpropagateSum}{Node, Value}
            \State Node.getValues().CumulativeUtility += Value
            \State Node.getValues().Count += 1
            \State Parent $\gets$ \textproc{GetParent}(Node)
            \If{Parent}
                \State \textproc{BackpropagateSum}(Parent, Value)
            \EndIf
        \EndProcedure
    \end{algorithmic}    
\end{algorithm}

\begin{algorithm}[H]
    \begin{algorithmic}[1]
        \Procedure{StoreStaticBackpropagateSum}{Node, Value}
            \State Evaluation, Simulation $\gets$ Value
            \Comment Requires Value to be 2-tuple
            \State Node.getValues().Evaluation $\gets$ Evaluation
            \State \textproc{BackpropagateSum}(Node, Simulation)
        \EndProcedure
    \end{algorithmic}    
\end{algorithm}

\begin{algorithm}[H]
    \begin{algorithmic}[1]
        \Procedure{BackpropagateSumAndMiniMax}{Node, Value}
            \State Evaluation, Simulation $\gets$ Value
            \Comment Requires Value to be 2-tuple
            \State \textproc{BackpropagateSum}(Node, Simulation)
            \State \textproc{BackpropagateMiniMax}(Node, Evaluation)
        \EndProcedure
    \end{algorithmic}    
\end{algorithm}

\newpage
\subsection*{Trimming}

\begin{algorithm}[H]
    \begin{algorithmic}[1]
        \Procedure{StoreMoveResetSearchTree}{Root, Frontier}
            \State Action $\gets$ \textproc{MiniMaxMove}(Root)
            \State Parameters.CurrentBestMove $\gets$ Action
            \State Parameters.MaxDepth += 1
            \For{C $\in$ \textproc{GetChildren}(Root)}
                \textproc{DelChild}(Root, C)
            \EndFor
            \State Root.OnFrontier $\gets$ True
        \EndProcedure
    \end{algorithmic}    
\end{algorithm}

\begin{algorithm}[H]
    \begin{algorithmic}[1]
        \Procedure{FractionPruning}{Root, Frontier}
            \State BranchingFactor $\gets$ Length(Root.getState().ApplicableActions)
            \State LowerBound $\gets$ Root.getValues().Count / (BranchingFactor + Parameters.PruningFactor) 
            \For{C $\in$ \textproc{GetChildren}(Root)}
                \If{C.getValues().Count $<$ LowerBound}
                    \State \textproc{DelChild}(Root, C)
                \ElsIf{IsFullyExpanded(C)}
                    \State \textproc{FractionPruning}(C, Frontier)
                \EndIf
            \EndFor
            
        \EndProcedure
    \end{algorithmic}    
\end{algorithm}


\newpage
\subsection*{Get Best Move}
\begin{algorithm}[H]
    \begin{algorithmic}[1]
        \Procedure{MiniMaxMove}{Root}
            \If{IsMaxNode(Root)}
                \State BestChild $\gets$ argmax\{V.Evaluation \textbf{for} V $\in$ Root.getChildrenValues()\}
            \Else
                \State BestChild $\gets$ argmin\{V.Evaluation \textbf{for} V $\in$ Root.getChildrenValues()\}
            \EndIf
            \State \Return BestChild.getGeneratingAction()
        \EndProcedure
    \end{algorithmic}    
\end{algorithm}

\begin{algorithm}[H]
    \begin{algorithmic}[1]
        \Procedure{GetStoredMove}{Root}
            \State \Return Parameters.CurrentBestMove 
        \EndProcedure
    \end{algorithmic}    
\end{algorithm}

\begin{algorithm}[H]
    \begin{algorithmic}[1]
        \Procedure{MostRobustChild}{Root}
            \State BestChild $\gets$ argmax\{V.Count \textbf{for} V $\in$ Root.getChildrenValues()\}
            \State \Return BestChild.getGeneratingAction() 
        \EndProcedure
    \end{algorithmic}    
\end{algorithm}

\begin{algorithm}[H]
    \begin{algorithmic}[1]
        \Procedure{WeightedUtilityMove}{Root}
            \State Max $\gets$ 0
            \State MaximizingAction $\gets$ None
            \For{V $\in$ Root.getChildrenValues()}
                \If{IsMaxNode(Root)}
                    \State Eval $\gets$ V.Evaluation
                    \State Estimate $\gets$ V.CumulativeUtility / V.Count
                \Else 
                    \State Eval $\gets$ 1 - V.Evaluation
                    \State Estimate $\gets$ 1 - V.CumulativeUtility / V.Count
                \EndIf
                \State Q $\gets$ 1 / $\sqrt{\text{V.Count}}$
                \State W $\gets$ Eval $\cdot$ Q + Estimate $\cdot$ (1 - Q)
                \If{W $>$ Max}
                    \State Max $\gets$ W
                    \State MaximizingAction $\gets$ V.GeneratingAction
                \EndIf
            \EndFor 
            \State \Return MaximizingAction
        \EndProcedure
    \end{algorithmic}    
\end{algorithm}

\newpage
\subsection*{Control Flow}
\begin{algorithm}[H]
    \begin{algorithmic}[1]
        \Procedure{WhenRootHasValue}{Root}
            \State \Return Root.getValues().Evaluation != None 
        \EndProcedure
    \end{algorithmic}    
\end{algorithm}

\begin{algorithm}[H]
    \begin{algorithmic}[1]
        \Procedure{WhenBudgetExceeded}{Root}
            \State \Comment Can either be time based or iteration based, but must return True when the time / iteration exceeds the budget. 
        \EndProcedure
    \end{algorithmic}    
\end{algorithm}

\begin{algorithm}[H]
    \begin{algorithmic}[1]
        \Procedure{IfStateIsTerminal}{Node, Value}
            \State \Return Node.getState() $\in S^\circ$  
        \EndProcedure
    \end{algorithmic}    
\end{algorithm}

\begin{algorithm}[H]
    \begin{algorithmic}[1]
        \Procedure{IfDepthReached}{Node, Value}
            \State \Return Node.getValues().Depth == Parameters.MaxDepth 
        \EndProcedure
    \end{algorithmic}    
\end{algorithm}

\begin{algorithm}[H]
    \begin{algorithmic}[1]
        \Procedure{IfDepthReachedOrFullyExpanded}{Node, Value}
            \State DepthReached $\gets$ Node.getValues().Depth == Parameters.MaxDepth
            \State FullyExpanded $\gets$ Length(Node.getValues().UnexpandedActions) == 0
            \State \Return DepthReached \textbf{or} FullyExpanded
        \EndProcedure
    \end{algorithmic}    
\end{algorithm}

\begin{algorithm}[H]
    \begin{algorithmic}[1]
        \Procedure{WhenPeriodic}{Root}
            \State \Return Root.getValues().Count \textbf{mod} Parameters.PeriodLength == 0
        \EndProcedure
    \end{algorithmic}    
\end{algorithm}

\newpage
\section{Experiment Results}
\label{appendix:results}

\begin{figure}[H]
    \centering
    \includegraphics[width=\textwidth]{images/connectfour_results.pdf}
    \caption{Winrates for the Max player in Connect Four, for the MiniMax- and MCTS-groups.}
    \label{fig:c4_results}
\end{figure}

\begin{figure}[H]
    \centering
    \includegraphics[width=\textwidth]{images/connectfour_results_folded.pdf}
    \caption{Combined Max- and Min-player winrates for the agents in Connect Four, for the MiniMax- and MCTS-groups.}
    \label{fig:c4_results_folded}
\end{figure}

\begin{figure}[H]
    \centering
    \includegraphics[width=.49\textwidth]{images/connectfour_top_four_results.pdf}
    \includegraphics[width=.49\textwidth]{images/connectfour_top_four_results_folded.pdf}
    \caption{Max-player (left) and combined (right) winrates for the top four agents, playing Connect Four.}
    \label{fig:c4_top_four_results}
\end{figure}

\begin{figure}[H]
    \centering
    \includegraphics[width=.49\textwidth]{images/nim_top_four_results.pdf}
    \includegraphics[width=.49\textwidth]{images/nim_top_four_results_folded.pdf}
    \caption{Max-player (left) and combined (right) winrates for the top four agents, playing Nim.}
    \label{fig:nim_top_four_results}
\end{figure}
