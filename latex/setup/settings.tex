\usepackage[english]{babel}
\usepackage{booktabs}
\usepackage[normalem]{ulem}
\usepackage{latexsym}
\usepackage{pdfpages}
\setlength{\parindent}{0pt}
\usepackage{setspace}
\usepackage{float}
\onehalfspacing
\usepackage{csquotes}
\usepackage{color}
\usepackage{afterpage}

\usepackage{algorithm}
\usepackage[noend]{algpseudocode}

% LISTINGS SETTINGS
\usepackage{listings}

\definecolor{background}{RGB}{39, 40, 34}
\definecolor{string}{RGB}{230, 219, 116}
\definecolor{comment}{RGB}{117, 113, 94}
\definecolor{normal}{RGB}{248, 248, 242}
\definecolor{identifier}{RGB}{166, 226, 46}

\lstnewenvironment{Python}
  {\lstset{
  language=python,                			% choose the language of the code
  numbers=left,                   		% where to put the line-numbers
  stepnumber=1,                   		% the step between two line-numbers.        
  numbersep=5pt,                  		% how far the line-numbers are from the code
  numberstyle=\tiny\color{black}\ttfamily,
  backgroundcolor=\color{background},  		% choose the background color. You must add \usepackage{color}
  showspaces=false,               		% show spaces adding particular underscores
  showstringspaces=false,         		% underline spaces within strings
  showtabs=false,                 		% show tabs within strings adding particular underscores
  tabsize=4,                      		% sets default tabsize to 2 spaces
  captionpos=b,                   		% sets the caption-position to bottom
  breaklines=true,                		% sets automatic line breaking
  breakatwhitespace=true,         		% sets if automatic breaks should only happen at whitespace
  title=\lstname,                 		% show the filename of files included with \lstinputlisting;
  basicstyle=\color{normal}\ttfamily,					% sets font style for the code
  keywordstyle=\color{magenta}\ttfamily,	% sets color for keywords
  stringstyle=\color{string}\ttfamily,		% sets color for strings
  commentstyle=\color{comment}\ttfamily,	% sets color for comments
  emph={format_string, eff_ana_bf, permute, eff_ana_btr},
  emphstyle=\color{identifier}\ttfamily
}}{}
 
\lstset{ % General setup for the package
    basicstyle=\small\sffamily,
    numbers=left,
    numberstyle=\small,
    frame=tb,
    tabsize=4,
    columns=fixed,
    showstringspaces=false,
    showtabs=false,
    keepspaces,
}


\setcounter{secnumdepth}{2}

% Specify the margins. This is 6.25inches in text with which 
% can be used to size figures to the correct size.
\usepackage[a4paper, margin=2.5625cm]{geometry}

\usepackage[parfill]{parskip}			% New line instead of indent for sections

%make math look very pretty:
\usepackage{amsmath}                    % Make pretty math
\usepackage{amssymb}                    % Extended symbol collection

%bibliography:
%\usepackage[citestyle=ieee, bibstyle=ieee,style=numeric-comp,sorting=nty,maxbibnames=99]{biblatex}
%\usepackage[citestyle=authortitle,bibstyle=numeric]{biblatex}
%Makes the last name first in the bibliography.
%\DeclareNameAlias{author}{last-first}
%\DeclareNameAlias{author}{family-given}
%\addbibresource{references.bib}
%layout:
\usepackage{tikz}
\usetikzlibrary{fadings}
\usetikzlibrary{calc}
\usetikzlibrary{decorations.pathmorphing}
\usetikzlibrary{shapes.geometric, arrows}
\usetikzlibrary{trees}
\usetikzlibrary{positioning}

\usepackage{eso-pic}					% Packages for layout and graphics (example set image as background image)
\usepackage{graphicx}
\usepackage{fancyhdr}					% Setting the style for header and footer.
\usepackage{lastpage}

\usepackage[hidelinks]{hyperref}		% Clickable links

\usepackage{caption}                % For subfigures
\usepackage{subcaption}
\usepackage[printonlyused]{acronym}
\usepackage{xspace}

\newcommand{\revise}[0]{{\colorbox{red}{REVISE}}}
\newcommand{\needcit}[0]{{\colorbox{red}{CITATION NEEDED}}}
\newcommand{\todo}[0]{{\colorbox{red}{TODO}}}

\newcommand{\shterm}[0]{\textproc{ShouldTerminate}\xspace}
\newcommand{\select}[0]{\textproc{Select}\xspace}
\newcommand{\expand}[0]{\textproc{Expand}\xspace}
\newcommand{\sheval}[0]{\textproc{ShouldEvaluate}\xspace}
\newcommand{\eval}[0]{\textproc{Evaluate}\xspace}
\newcommand{\shbp}[0]{\textproc{ShouldBackpropagate}\xspace}
\newcommand{\bp}[0]{\textproc{Backpropagate}\xspace}
\newcommand{\shtrim}[0]{\textproc{ShouldTrim}\xspace}
\newcommand{\trim}[0]{\textproc{Trim}\xspace}
\newcommand{\gbm}[0]{\textproc{GetBestMove}\xspace}


\algdef{SE}% flags used internally to indicate we're defining a new block statement
[STRUCT]% new block type, not to be confused with loops or if-statements
{Obj}% "\Struct{name}" will indicate the start of the struct declaration
{EndObj}% "\EndStruct" ends the block indent
[1]% There is one argument, which is the name of the data structure
{\textbf{object} \textsc{#1}}% typesetting of the start of a struct
{\textbf{end object}}% typesetting the end of the struct

\algdef{SE}% flags used internally to indicate we're defining a new block statement
[GSTRUCT]% new block type, not to be confused with loops or if-statements
{GObj}% "\Struct{name}" will indicate the start of the struct declaration
{EndGObj}% "\EndStruct" ends the block indent
[1]% There is one argument, which is the name of the data structure
{\textbf{global object} \textsc{#1}}% typesetting of the start of a struct
{\textbf{end object}}% typesetting the end of the struct



% Styling the header and footer
\fancyhf{}
\fancyhead{}
\fancyfoot{}
\fancyhead[L]{\fontsize{11}{10}\selectfont\leftmark}
\fancyfoot[R]{\thepage \hspace{1pt} af \pageref{LastPage}}
\setlength{\headheight}{15.5pt}

\fancypagestyle{plain}{
    \fancyhf{}
    \fancyhead{}
    \fancyfoot{}
    \renewcommand{\headrulewidth}{0pt}
}
\pagestyle{fancy}

% Making the command for placing text in random locations
\newcommand\PlaceText[3]{%
\begin{tikzpicture}[remember picture,overlay]
\node[outer sep=0pt,inner sep=0pt,anchor=south west] 
  at ([xshift=#1,yshift=-#2]current page.north west) {#3};
\end{tikzpicture}%
}

